\documentclass{article}

%-------------------------------------------------------------------------
\usepackage{romannum}
\usepackage{soul}
\usepackage{enumitem,multicol}
\usepackage[a4paper]{geometry}
\usepackage{color}
\usepackage{siunitx}
\usepackage{amsmath,amssymb}
\definecolor{light-gray}{gray}{0.85}
\definecolor{bl}{RGB}{173,216,230}
\newcommand{\hilight}[1]{\colorbox{light-gray}{#1}}
\geometry{top=.75in, bottom=.75in, left=.58in, right=.58in}
%-------------------------------------------------------------------------
%font san serif family
\sffamily
\upshape
\mdseries

%-------------------------------------------------------------------------
%paragraph spacings
\def\vsp{6pt}
\def\Plus{\texttt{+}}
\setlength{\parindent}{0pt}
\setlength{\parskip}{0pt}
\setlength{\parsep}{0pt}
%-------------------------------------------------------------------------

%-------------------------------------------------------------------------
%for horizontal lines
\usepackage[pdftex]{graphicx}
\newcommand{\HRule}{\rule{\linewidth}{0.2mm}}
%-------------------------------------------------------------------------
\begin{document}
%for iitb logo + marks
\vspace*{140pt}
%\underline{\makebox[\textwidth][l]{\bfseries %\color{black}MAJOR PROJECT}}
%\begin{itemize}[leftmargin=3.5mm]
%	\item 
%\end{itemize}
\underline{\makebox[\textwidth][l]{\bfseries \color{black} AREAS OF INTEREST}}\\
\vspace{0mm}

\vspace{-2.6mm}Embedded Systems, Energy Harvesting, Internet of Things\\


\vspace{-2mm}\underline{\makebox[\textwidth][l]{\bfseries \color{black}TECHNICAL SKILLS}}

\vspace{-2mm}\begin{itemize}[leftmargin=3.5mm]
	\setlength{\itemsep}{1mm}
	\item \textbf{Languages}: C/C++, Python, Bash, VHDL 
	\vspace{-2mm}
	\item\textbf{Tools \& IDEs}: Git, \LaTeX, Atmel Studio, Code Composer Studio, Eagle, Quartus, MATLAB 
\end{itemize}

\vspace{-2mm}\underline{\makebox[\textwidth][l]{\bfseries \color{black}MAJOR PROJECT AND SEMINAR}}
\begin{itemize}[leftmargin=3.5mm]
	\setlength{\itemsep}{1mm}
	\item \textbf{Design of IoT based Energy efficient subsystem for greenhouse} (\textit{M.Tech Project})\hfill\hspace{-20pt}{\textit{[Jun'17 - present]}}
	\\\textit{(Guide: Prof. Kavi Arya)}
	\vspace{-8pt}
	\renewcommand{\labelitemii}{--}
	\begin{itemize}[leftmargin=3.5mm]
		\setlength{\itemsep}{-1mm}
		\item \textbf{Idea}
		\renewcommand{\labelitemiii}{-}
		\vspace{-1mm}
		\begin{itemize}[leftmargin=3.5mm]
			\setlength{\itemsep}{0.4mm}
			\item A \textbf{closed loop} irrigation control system for urban farming to promote optimum growth.
			\item A low power sensor node with \textbf{solar energy harvesting} capability and an actuator for drip irrigation.
			\item Part of a low maintenance and affordable solution for sustainable urban farming.
		\end{itemize}
		\item{\textbf{Completed Work}}
		\begin{itemize}[leftmargin=3.5mm]
			\setlength{\itemsep}{0.4mm}
			\item Studied about solar harvesting power supply design and \textbf{duty cycling} for low power operation.
			\item Modified an existing \textbf{Wifi} based \textbf{solenoid valve} controller for single battery low power operation.
		\end{itemize}
		\item \textbf{Ongoing and Future Work}
		\begin{itemize}[leftmargin=3.5mm]	\setlength{\itemsep}{0.4mm}
			\item Design of solar harvesting power supply for a sensor node  and requirements for \textbf{energy neutrality}.
			\item \textbf{Real time monitoring} of soil moisture for closed loop control of irrigation.
			\item Analyzing an  \textbf{Evapotranspiration} estimation model and its effectiveness in irrigation scheduling.
		\end{itemize}
		
	\end{itemize}
\end{itemize}
\vspace{-4mm}
\begin{itemize}[leftmargin=3.5mm]
	\setlength{\itemsep}{-0.29em}
	\item \textbf{Study of Energy Harvesting for Embedded Systems} \textit{(Seminar)}\hfill\hspace{-20pt}{\textit{[Jan'17 - Apr'17]}}
	\\\textit{(Guide: Prof. Kavi Arya)}
	\vspace{-6pt}
	\begin{itemize}[leftmargin=3.5mm]
		\vspace{-0.25mm}
		\setlength{\itemsep}{0.4mm}
		\item Surveyed the different \textbf{ambient energy} sources available and their harvesting potential.
		\vspace{-0.5mm}
		\item Practically examined the \textbf{V-I characteristics} of 6V, 200 mA solar panel in different levels of illuminance. 
		\vspace{-0.5mm}
		\item Built a \textbf{data logging device} to measure the current output from the solar panel. 
	\end{itemize}
\end{itemize}

\vspace{-1mm}\underline{\makebox[\textwidth][l]{\bfseries \color{black}WORK EXPERIENCE}}

\vspace{2mm}\textbf{e-Yantra, Department of Computer Science \& Engineering, IIT Bombay} \hfill\textit{[February 2014 - present]}\\
\textit{Senior Project Technical Assistant}
\vspace{-2mm}
\begin{itemize}[leftmargin=3.5mm]
	\setlength{\itemsep}{-0.7mm}
	\item Conducted \textbf{9} two-day workshops covering the basics of an \textbf{Atmega2560} based Robotics and Embedded research platform for teachers of engineering and polytechnic colleges in different regions of the country. 
	\item Intergral part of the e-Yantra Lab Setup Initiative (eLSI) team, responsible for setting up Robotics and Embedded Systems labs in \textbf{208} colleges across the country.
	\item Conceptualized and implemented a module based online learning method \textbf{(Task Based Training)} for teachers on basics of Embedded systems along with another team member. 
	\item Created learning modules for Task Based Training and successfully \textbf{coordinated} with a team to complete \textbf{five} editions of this online training.
	\item Key member of a team involved in organizing and handling an annual \textbf{e-Yantra Symposium (eYS)}  having representation from \textbf{100\Plus} colleges for the last two years.
	\item Streamlined routine communication flows and data collection for interaction with engineering colleges. 
	\item Core member of a team that developed \textbf{Themes} (real-world problems abstracted into games) based on Valet Parking and Plant Growth Monitoring as challenges for teachers after completing Task Based Training. 
	\item Created a Fire Fighting Robot Theme in a team of three, for the national level e-Yantra Robotics Competition (eYRC) for students.
\end{itemize}
\underline{\makebox[\textwidth][l]{\bfseries \color{black}RELEVANT COURSES}}
\vspace{-5mm}
\begin{itemize}[leftmargin=3.5mm]
	\vspace{-0.5mm}
	\setlength{\itemsep}{1mm}
	\item \textbf{Embedded}: Electronics System Design, Embedded System Design, Sensors in Instrumentation, Software Development Techniques for Engineering \& Scientists
	\vspace{-2mm}
	\item\textbf{Digital Design}: System Design, VLSI Design Lab, Foundation of VLSI CAD (Ongoing)
	\vspace{-2mm}
	\item \textbf{Signal Processing}: Digital Signal Processing \& its Applications, Digital Signal Processing - System Design \& Implementation 
\end{itemize}
\underline{\makebox[\textwidth][l]{\bfseries \color{black}POSITIONS OF RESPONSIBILITY}}

\vspace{-2mm}\begin{itemize}[leftmargin=3.5mm]
	\setlength{\itemsep}{1mm}
	\item \textbf{Teaching Assistant} for Embedded Systems course of Department in Computer Science \& Engineering (CS 684) for Autumn Semester, 2016. Assisted in designing lab experiments on the TM4C123G Launchpad for the course.
		\vspace{-2mm}
	\item Mentor for student internship projects based on sensor interfacing, Internet of things application and \textbf{Unit testing} for Embedded C code.
		\vspace{-2mm}
	\item Member of the core team that \textbf{organized} the national level e-Yantra Robotics Competition finals in 2015 and 2016. 
\end{itemize}
\underline{\makebox[\textwidth][l]{\bfseries \color{black}COURSE PROJECTS}}

\vspace{-2mm}\begin{itemize}[leftmargin=3.5mm]
	\setlength{\itemsep}{-0.29em}
	\item \textbf{Air Quality Monitoring }\hfill{\textit{[Jan'17 - Apr'17]}}
	\\\textit{(Guide: Prof. Krithi Ramamritham)}
	\vspace{-6pt}
	\begin{itemize}[leftmargin=3.5mm]
		\vspace{-0.5mm}
		\setlength{\itemsep}{1mm}
		\item Designed a \textbf{MSP430F5529} based  sensor node having a \textbf{stackable} design with temperature, humidity, CO and particulate matter (PM 2.5) sensors on-board.
		\vspace{-1.5mm}
		\item PM 2.5 and CO sensor were calibrated using their sensitivity characteristics and the performance of low cost PM 2.5 sensor was \textbf{compared} with a commercially available sensor. 
	\end{itemize}
\end{itemize}
\vspace{-5mm}
\begin{itemize}[leftmargin=3.5mm]
	\setlength{\itemsep}{-0.29em}
	\item \textbf{Image Compression and Wavelets}\hspace{2mm}\hfill{\textit{[Jan'17 - Apr'17]}}
	\\\textit{(Guide: Prof. Sachin Patkar)}
	\vspace{-6pt}
	\begin{itemize}[leftmargin=3.5mm]
		\vspace{-0.5mm}
		\setlength{\itemsep}{1mm}
		\item Prototyped Wavelet based image compression in \textbf{MATLAB} and then implemented \textbf{2D Haar Wavelet Analysis} filter bank with thresholding in \textbf{VHDL}.
		\vspace{-1.5mm}
		\item Built a Nios-II based \textbf{Qsys} component on the DE0-Nano \textbf{FPGA} development platform for 1D Discrete Haar  Wavelet transform.
	\end{itemize}
\end{itemize}
\vspace{-5mm}
\begin{itemize}[leftmargin=3.5mm]
	\setlength{\itemsep}{-0.29em}
	\item \textbf{Python API for mobile robot control  }\hfill{\textit{[Jul'16 - Nov'16]}}
	\\\textit{(Guide: Prof. Prabhu Ramchandran)}
	\vspace{-6pt}
	\begin{itemize}[leftmargin=3.5mm]
		\vspace{-0.5mm}
		\setlength{\itemsep}{1mm}
		\item Developed a \textbf{Python API} along with the corresponding firmware to control a mobile robotic platform using \textbf{Raspberry Pi} providing an \textbf{abstraction} over Embedded C.
		\vspace{-1.5mm}
		\item The project involved following \textbf{coding guidelines} (PEP8), use of version control (Git), \textbf{documentation tools} (Sphinx) and \textbf{Unit testing }for Python Code.
	\end{itemize}
\end{itemize}
\vspace{-5mm}
\begin{itemize}[leftmargin=3.5mm]
	\setlength{\itemsep}{-0.29em}
	\item \textbf{Multiload Dimmer}\hspace{2mm}\hfill{\textit{[Jan'16 - Apr'16]}}
	\\\textit{(Guide: Prof. P. C. Pandey)}
	\vspace{-8pt}
	\begin{itemize}[leftmargin=3.5mm]
		\vspace{-0.5mm}
		\setlength{\itemsep}{1mm}
		\item Implemented a micro-controller based \textbf{power control} of multiple loads along with frequency compensation.  
		\vspace{-1.5mm}
		\item Supplemented the system with an Android app having ON/OFF, intensity and \textbf{intensity-duration} control.
	\end{itemize}
\end{itemize}
\vspace{-5mm}
\begin{itemize}[leftmargin=3.5mm]
	\setlength{\itemsep}{-0.29em}
	\item \textbf{Multiband Dynamic Range Compression for Hearing Aids}\hfill\hspace{-20pt}{\textit{[Jul'15 - Nov'15]}}
	\\\textit{(Guide: Prof. Vikram Gadre)}
	\vspace{-8pt}
	\begin{itemize}[leftmargin=3.5mm]
		\vspace{-0.5mm}
		\setlength{\itemsep}{1mm}
		\item Built a frequency dependent gain function based on \textbf{FFT Analysis} and \textbf{Synthesis }for auditory critical bands.
		\vspace{-1.5mm}
		\item The proposed solution was successfully tested on \textbf{TMS320C5515} Digital Signal Processor using a pre-recorded sentence.
	\end{itemize}
\end{itemize}
\vspace{-5mm}
\begin{itemize}[leftmargin=3.5mm]
	\setlength{\itemsep}{-0.29em}
	\item \textbf{Auto-zeroing Differential Amplifier   }\hfill{\textit{[Jul'15 - Nov'15]}}
	\\\textit{(Guide: Prof. P. C. Pandey)}
	\vspace{-8pt}
	\begin{itemize}[leftmargin=3.5mm]
		\vspace{-0.5mm}
		\setlength{\itemsep}{1mm}
		\item Designed a \textbf{reset stabilized amplifier} using an internal ADC of a micro-controller for sampling and a serially controlled DAC to generate the compensation voltage for offset nulling.
		\vspace{-1.5mm}
		\item Tested the solution with a differential amplifier having a gain of 100 built using Op-amp IC \SI{}{\micro\A}741.
	\end{itemize}
\end{itemize}




\underline{\makebox[\textwidth][l]{\bfseries \color{black}OTHER ACTIVITIES}}
\begin{itemize}[leftmargin=3.5mm]
	\vspace{-0.5mm}
	\setlength{\itemsep}{1mm}
	\item Enjoy playing Squash and Cricket.
	\vspace{-1.5mm}
	\item Other hobbies include watching Standup Comedy and Squash tournament matches online.  
\end{itemize}
\end{document}